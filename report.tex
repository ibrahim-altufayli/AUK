% --------------------------------------------------------------------------
% Template for ICAD-2024 paper; to be used with:
%          icad2024.sty  - ICAD 2024 LaTeX style file, and 2024
%          IEEEbtran.bst - IEEE bibliography style file.
%
% --------------------------------------------------------------------------
\UseRawInputEncoding
\documentclass[a4paper,10pt,oneside]{article}
\usepackage[utf8]{inputenc}
\usepackage{icad2024,amsmath,epsfig,times,url,hyperref}
\usepackage[T1]{fontenc}

% Example definitions.
% --------------------


% Title.
% --------------------
\title{K-Nearest Neighbours Laboratory}

% *** IMPORTANT ***
% *** PLEASE LEAVE AUTHOR INFORMATION BLANK UNTIL FINAL CAMERA-READY SUBMISSION *** 

% IF ONE AUTHOR
%\name{Jyri Huopaniemi} 
%\address{Nokia Research Center \\ 
%Speech and Audio Systems Laboratory \\ 
%P.O.Box 407, FIN-00045 Nokia Group, Finland \\ 
%{\tt jyri.huopaniemi@nokia.com}} 
%

% IF TWO AUTHORS
\twoauthors
{Gabriele Francesco Berruti} {University of Genoa \\ Via Roma 37/6 \\ Genova, Italy  \\ {\tt gabriele.berruti@unige.edu}}
{Ibrahim Altufayli} {University of Genoa \\ Via Roma 37/6 \\ Genova, Italy  \\ {\tt ibrahim.altufayli@unige.edu}}

\begin{document}
\ninept
\maketitle
\begin{sloppy}
\begin{abstract}
This is the template file for the proceedings of the 2024 International Conference on Auditory Display, which will be held June 24 to 28, 2024 at Rensselaer Polytechnic Institute, Troy, NY, USA. This template has been generated from the ICAD 2024 template and aims at producing conference proceedings in electronic form. Please use either \LaTeX\ or Microsoft Word formats when preparing your submission. All questions concerning ICAD 2024 submissions should be addressed to the paper chairs (\href{mailto:papers@icad2024.icad.org}{papers@icad2024.icad.org}). These guidelines and templates are adapted from previous editions of ICAD conferences, so an experienced author who has published something in these proceedings will find them familiar. The templates are available in electronic form at the website: \url{https://icad2024.icad.org/}\\
\bf{The length of the abstract should be approximately 150 words.}
\end{abstract}

\section{Introduction}
\label{sec:intro}

Classification may be defined as the process of predicting class or category from observed values or given data points. The categorized output can have the form such as “Black” or “White” or “spam” or “no spam”.
There are different classification methods that could be used to perform the task in different contexts and one amoung the used methods are  
\section{Formatting your paper}
\label{sec:format}

All manuscripts must be formatted for white A4 paper (8.27 in x 11.02 in). Please do \textbf{not} use US letter-size papers. All printed material, including text, illustrations, and charts, must be kept within a print area of 6.27 inches (169 mm) wide by 9.2 inches (233 mm) high. Do not write or print anything outside the print area. The top margin must be 1 inch (25 mm), except for the title page, and the left margin must be 0.75 inch (19 mm). All \textit{text} must be in a two-column format. Columns are to be 3.29 inches (83.5 mm) wide, with a 0.31 inch (8 mm) space between them. Text must be fully justified.

\section{Page Title Section}
\label{sec:pagestyle}

The paper title (on the first page) should begin 0.98 inches (25 mm) from the top edge of the page, centered, completely capitalized, and in Times New Roman 14-point, boldface type. The authors' name(s) and affiliation(s) appear below the title. Papers with multiple authors and affiliations may require a greater amount of space for this information.

\section{Type-Style and Fonts}
\label{sec:typestyle}

To achieve the best rendering in the proceedings, we strongly encourage you to use Times New Roman font. In addition, this will give the proceedings a more uniform look. Use a font size that is no smaller than nine point type throughout the paper, including figure captions.

In nine point type font, capital letters are 2 mm high. {\bf If you use the smallest point size, there should be no more than 3.2 lines/cm (8 lines/inch) vertically.} This is a minimum spacing; 2.75 lines/cm (7 lines/inch) will make the paper much more readable. Larger type sizes require correspondingly larger vertical spacing. Please do not double-space your paper. True-Type 1 fonts are preferred.

The first paragraph in each section should not be indented, but all the following paragraphs within the section should be indented as these paragraphs demonstrate.

\section{Major Headings}
\label{sec:majhead}

Major headings, as for this section, should appear in all capital letters, bold face, centered in the column, with one blank line before, and one blank line after. Use a period (``.'') after the heading number, not a colon.

\subsection{Subheadings}
\label{ssec:subhead}

Subheadings should appear in sentence case and in boldface. They should start at the left margin on a separate line. 
 
\subsubsection{Sub-subheadings}
\label{sssec:subsubhead}

Sub-subheadings, as for this paragraph, are discouraged. However, if you must use them, they should appear in they should appear in sentence case and start at the left margin on a separate line, with paragraph text beginning on the following line. They should be in italics.  \\

\section{Page Numbering, Header, and Footer}
\label{sec:page}

\underline{Please do {\bf not} paginate your manuscript.} Page numbers, session numbers, and conference identification will be inserted when the manuscript is included in the proceedings. In addition, please do {\bf not} change or remove the header and footer.

\section{ILLUSTRATIONS, GRAPHS, AND PHOTOGRAPHS}
\label{sec:illust}

Illustrations must appear within the designated margins. They may span the two columns. If possible, position illustrations at the top of columns rather than in the middle or at the bottom. Caption and number every illustration. All halftone illustrations must be clear black and white prints. Colors may be used, but they should be selected so as to be readable when printed on a black and white-only printer.

Since there are many ways, often incompatible, of including images (e.g., with experimental results) in a \LaTeX ~document, an example of how to do this is presented in Fig.~\ref{fig:results}.

% Below is an example of how to insert images. Delete the ``\vspace'' line,
% uncomment the preceding line ``\centerline...'' and replace ``imageX.ps''
% with a suitable PostScript file name.
% -------------------------------------------------------------------------


\section{Equations}
\label{sec:equations}

Equations should be placed on separate lines and consecutively numbered with equation numbers in parentheses flush with the right margin, as illustrated in (\ref{eqn:wave_equation}), which gives the homogeneous acoustic wave equation in Cartesian coordinates \cite{eWilliams1999},
\begin{equation}
  \label{eqn:wave_equation}
    \nabla^2p(x,y,z,t)-
    \displaystyle\frac{1}{c^2}\frac{\partial^2p(x,y,z,t)}{\partial t^2}=0,
\end{equation}
where $p(x,y,z,t)$ is an infinitesimal variation of acoustic pressure from its equilibrium value at position $(x,y,z)$ and time $t$, and where $c$ denotes the speed of sound.

Symbols in your equation should be defined before the equation appears or immediately following it. Use (1), not Eq. (1) or equation (1), except at the beginning of a sentence: ``Equation (1) is ...''

\section{FOOTNOTES}
\label{sec:foot}

To help your readers, avoid using footnotes and instead include necessary peripheral observations in the text (within parentheses, if you prefer, as in this sentence). Footnotes that do appear should be placed at the bottom of the column on the page on which they are referenced. Use Times New Roman 9-point type, single-spaced. 

\section{REFERENCES}
\label{sec:ref}

List and number all bibliographical references at the end of the paper. The references should be numbered in order of appearance in the document. When referring to them in the text, place the corresponding reference number in square brackets as shown at the end of this sentence \cite{aBee2001}, \cite{mSmith2001}. For \LaTeX\ users, the use of the Bib\TeX\ style file IEEEtran.bst is recommended, as included in this template.

\section{ACKNOWLEDGMENT}
\label{sec:ack}

The preferred spelling of the word acknowledgment in America is without an ``e'' after the ``g''. Try to avoid the stilted expression, ``One of us (R. B. G.) thanks ...''. Instead, try ``R.B.G.\ thanks ...''. Put sponsor acknowledgments in the unnumbered footnote on the first page.

% -------------------------------------------------------------------------
% Either list references using the bibliography style file IEEEtran.bst
\bibliographystyle{IEEEtran}
\bibliography{refs2024}
%
% or list them by yourself
% \begin{thebibliography}{9}
% 
% \bibitem{icad2015web}
%   \url{http://www.icad.org}.
%
%\bibitem[1]{icad1} A.~Bee, C.D.~Player, and X.~Lastname, ``A correct citation,'' in {\it Proc. of the 1st Int. Conf. (IC)}, Helsinki, Finland, June 2001, pp. 1119-1134.  
%\bibitem[2]{icad2} E.~Zwicker and H.~Fastl, {\it Psychoacoustics: Facts and Models}, Springer-Verlag, Heidelberg, Germany, 1990.
%\bibitem[3]{icad3} M.R.~Smith, ``A good journal article,'' {\it J. Acoust. Soc. Am.}, vol. 110, no. 3, pp. 1598--1608, Mar. 2001.
% 
% \end{thebibliography}

\end{sloppy}
\end{document}
